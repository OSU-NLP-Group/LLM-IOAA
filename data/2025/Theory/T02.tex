\documentclass[10pt]{article}
\usepackage[utf8]{inputenc}
\usepackage[T1]{fontenc}
\usepackage{amsmath}
\usepackage{amsfonts}
\usepackage{amssymb}
\usepackage[version=4]{mhchem}
\usepackage{stmaryrd}
\usepackage{graphicx}
\usepackage[export]{adjustbox}
\graphicspath{ {./images/} }

\begin{document}

    The festival of "Makar-Sankranti" is celebrated in India when the Sun appears to enter the zodiacal region of Capricorn $($ Makar $=$ Capricorn, Sankranti $=$ Entry) as seen from the Earth. It is currently celebrated around 14 January every year. Many years ago this festival also coincided with the Winter Solstice in the northern hemisphere which we assume to take place on 21 December.\\
    (T02.1) Based on the information above, find the year, $y_{\mathrm{c}}$, when the celebration of this festival last coincided with the Winter Solstice in the Northern hemisphere.\\
    (T02.2) If the Sun appeared to enter the zodiacal region of Capricorn at a local time of 11:50:13 hrs on 14 January 2006 in Mumbai, calculate the date, $D_{\text {enter }}$, and local time, $t_{\text {enter }}$, of its entry in Capricorn in the year 2013.\\
    (T02.3) Makar-Sankranti festival is celebrated at a given place on the day of the first sunset in the zodiacal region of Capricorn. You may assume that the local sunset time for Mumbai in January is 18:30:00 hrs.

    Indicate the date of celebration of the festival on every year between 2006 and 2013.


\end{document}