\documentclass[10pt]{article}
\usepackage[utf8]{inputenc}
\usepackage[T1]{fontenc}
\usepackage{amsmath}
\usepackage{amsfonts}
\usepackage{amssymb}
\usepackage[version=4]{mhchem}
\usepackage{stmaryrd}
\usepackage{graphicx}
\usepackage[export]{adjustbox}
\graphicspath{ {./images/} }

\begin{document}

    Orbiting binary black holes generate gravitational waves. Consider two black holes, in our Galaxy with masses $M=36 \mathrm{M}_{\odot}$ and $m=29 \mathrm{M}_{\odot}$, revolving in circular orbits with orbital angular frequency $\omega$ around their centre of mass.\\
    (T03.1) Assuming Newtonian gravity, derive an expression for the angular frequency, $\omega_{\mathrm{ini}}$, of the black hole orbits at a time, $t_{\text {ini }}$, when the separation between the black holes was 4.0 times the sum of their Schwarzschild radii, in terms of only $M, m$, and physical constants.
    
    Calculate the value of $\omega_{\text {ini }}\left(\right.$ in $\left.\mathrm{rad} \mathrm{s}^{-1}\right)$.\\
    (T03.2) In general relativity, black holes in orbit emit gravitational waves with frequency $f_{\mathrm{GW}}$, such that $2 \pi f_{\mathrm{GW}}=\omega_{\mathrm{GW}}=2 \omega$. This shrinks the black hole orbits, which in turn increases $f_{\mathrm{GW}}$. The rate of change of $f_{\text {GW }}$ is
    
    $$
    \frac{d f_{\mathrm{GW}}}{d t}=\frac{96 \pi^{8 / 3}}{5} G^{5 / 3} c^{\beta} M_{\mathrm{chirp}}{ }^{\alpha / 3} f_{\mathrm{GW}}^{\delta / 3},
    $$
    
    where $M_{\text {chirp }}=\frac{(m M)^{3 / 5}}{(m+M)^{1 / 5}}$ is called the "chirp mass".\\
    Find the values of $\alpha, \beta$ and $\delta$.\\
    (T03.3) Assume that the gravitational waves associated with the event were first detected at time $t_{\mathrm{ini}}=0$.\\
    Derive an expression for the observed time of black hole merger, $t_{\text {merge }}$, when $f_{\text {GW }}$ becomes very large, in terms of $\omega_{\text {ini }}, M_{\text {chirp }}$, and physical constants only. Calculate the value of $t_{\text {merge }}$ (in seconds).


\end{document}