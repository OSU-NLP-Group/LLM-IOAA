\documentclass[10pt]{article}
\usepackage[utf8]{inputenc}
\usepackage[T1]{fontenc}
\usepackage{amsmath}
\usepackage{amsfonts}
\usepackage{amssymb}
\usepackage[version=4]{mhchem}
\usepackage{stmaryrd}
\usepackage{graphicx}
\usepackage[export]{adjustbox}
\graphicspath{ {./images/} }
\usepackage{caption}

%New command to display footnote whose markers will always be hidden
\let\svthefootnote\thefootnote
\newcommand\blfootnotetext[1]{%
  \let\thefootnote\relax\footnote{#1}%
  \addtocounter{footnote}{-1}%
  \let\thefootnote\svthefootnote%
}

%Overriding the \footnotetext command to hide the marker if its value is `0`
\let\svfootnotetext\footnotetext
\renewcommand\footnotetext[2][?]{%
  \if\relax#1\relax%
    \ifnum\value{footnote}=0\blfootnotetext{#2}\else\svfootnotetext{#2}\fi%
  \else%
    \if?#1\ifnum\value{footnote}=0\blfootnotetext{#2}\else\svfootnotetext{#2}\fi%
    \else\svfootnotetext[#1]{#2}\fi%
  \fi
}

\begin{document}

\section*{6 Photometry of Binary stars (20 Points)}
We observe a binary system, at a distance $d=89 \mathrm{pc}$, with a circular, edge-on orbit around the common center of mass and period of 100 days. Our space-based telescope is of diameter $D=10 \mathrm{~m}$ and operates at a wavelength $\lambda=364 \mathrm{~nm}$. We notice that for a total of 38 days during each full period, the two stars cannot be resolved by this telescope as separate objects. The wavelength of peak emission for star $A$ is $\lambda_{A}=500 \mathrm{~nm}$ and that for star $B$ is $\lambda_{B}=600 \mathrm{~nm}$.\\
(a) What are the temperatures of the stars $A$ and $B,\left(T_{A}, T_{B}\right)$\\
(b) Calculate the distance $l$ between the stars.\\
(c) Calculate the sum of the masses of the stars $\left(M_{T}\right)$.

Combined photometry of the system:

\begin{center}
\begin{tabular}{llcccc}
 & Configuration & $U_{0}$ & $(U-B)$ & $(B-V)$ & $B C$ \\
1 & Stars next to each other & 6.39 & 0.2 & 0.1 & 0.1 \\
2 & B transiting in front of A & 6.86 & 0.25 & 0.12 & 0.175 \\
\end{tabular}
\end{center}

The interstellar extinction per kpc of $U$ is $a_{U}=1.4 \mathrm{mag} / \mathrm{kpc}$. The ratio of the densities of the stars $\rho_{A} / \rho_{B}=0.7$\\
note: For Bolometric correction we use convention:

$$
B C=m_{b o l}-m_{V}
$$

(d) Calculate the masses of both the stars $\left(M_{A}, M_{B}\right)$.

\end{document}