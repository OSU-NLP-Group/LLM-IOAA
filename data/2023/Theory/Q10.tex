\documentclass[10pt]{article}
\usepackage[utf8]{inputenc}
\usepackage[T1]{fontenc}
\usepackage{amsmath}
\usepackage{amsfonts}
\usepackage{amssymb}
\usepackage[version=4]{mhchem}
\usepackage{stmaryrd}
\usepackage{caption}
\usepackage{multirow}

\DeclareUnicodeCharacter{00D7}{\ifmmode\times\else{$\times$}\fi}

\begin{document}

\section*{Theory 10: 'Aldebaran'}
On 9 March 1497, Nicolaus Copernicus observed the occultation of Aldebaran by the Moon from Bologna. In his work De revolutionibus orbium cœlestium ${ }^{1}$ Copernicus described the event: "I saw the star touching the dark edge of the Moon and disappearing at the end of the 5th hour of the night between the horns of the Moon, closer to the south horn by a third of the Moon's diameter."

Assuming that the occultation was observed on the local meridian, that at maximum occultation Aldebaran was $0.32^{\prime}$ above the southern edge of the Moon, and that the apparent angular diameter of the Moon as seen from Bologna was 31.5', solve the following tasks:\\
(a) Find the latitude $\varphi_{1}$ of a place with the same longitude as Bologna, from which Aldebaran would have appeared to pass behind the centre of the Moon.\\
(b) Find the duration of the occultation as seen from latitude $\varphi_{1}$ if Aldebaran appeared to pass along the diameter of the lunar disk. For simplicity, also assume that the Moon and the observer are moving linearly at constant speed, that the Moon's orbit is circular and that the declination of the Moon does not change during the occultation.\\
(c) Find the topocentric angular velocity of the Moon during the occultation for an observer at latitude $\varphi_{1}$, in arcmin/hour, applying the same assumptions as in part (b).\\
(d) Estimate the range of the Moon's topocentric angular velocities in arcmin/hour for an observer at latitude $\varphi_{1}$, assuming a circular orbit. Show how this result can be justified by expressing the relative velocity of the Moon and observer in terms of their velocity vectors.

The declination of Aldebaran was $\delta_{\mathrm{A}}=15.37^{\circ}$ in 1497 (due to precession), and the latitude of Bologna is $\varphi_{\mathrm{B}}=44.44^{\circ} \mathrm{N}$.\\
(25 points)


\end{document}