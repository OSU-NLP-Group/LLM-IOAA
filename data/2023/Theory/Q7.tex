\documentclass[10pt]{article}
\usepackage[utf8]{inputenc}
\usepackage[T1]{fontenc}
\usepackage{amsmath}
\usepackage{amsfonts}
\usepackage{amssymb}
\usepackage[version=4]{mhchem}
\usepackage{stmaryrd}
\usepackage{caption}
\usepackage{multirow}

\DeclareUnicodeCharacter{00D7}{\ifmmode\times\else{$\times$}\fi}

\begin{document}

\section*{Theory 7: 'Libration'}
As a result of libration, studied among others by Johannes Hevelius, more than half of the Moon's surface can be observed from Earth.\\
(a) Estimate the maximum angle of libration in latitude $\phi_{B}$. The axial tilt (obliquity) of the Moon with respect to its orbital plane is $\alpha=6^{\circ} 41^{\prime}$.\\
(b) Estimate the maximum angle of libration in longitude $\phi_{L}$. Assume that the Moon is always aligned with the same side facing towards the second focus F2 of its orbit, and that the eccentricity of the Moon's orbit $e$ changes between 0.044 and 0.064 on a timescale of several months.\\
(c) Determine the fraction of the Moon's surface which can be seen from Earth.\\
(d) Calculate how many months (lunations) are needed for an observer to see the entirety of the fraction of the Moon's surface determined in part (c).\\

\end{document}