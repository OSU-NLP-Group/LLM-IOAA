\documentclass[10pt]{article}
\usepackage[utf8]{inputenc}
\usepackage[T1]{fontenc}
\usepackage{amsmath}
\usepackage{amsfonts}
\usepackage{amssymb}
\usepackage[version=4]{mhchem}
\usepackage{stmaryrd}
\usepackage{caption}
\usepackage{multirow}

\DeclareUnicodeCharacter{00D7}{\ifmmode\times\else{$\times$}\fi}

\begin{document}

\section*{Theory 5: 'Dark Energy'}
Observations indicate that the expansion of the Universe is accelerating. Fluctuations of the cosmic microwave background favour a flat (Euclidean) geometry, in which the total mass density (i.e. density of matter and equivalent mass density of all forms of energy) should be equal to the so-called critical density:

$$
\rho_{\mathrm{cr}}=\frac{3 H_{0}^{2}}{8 \pi G},
$$

where $H_{0}$ is the present value of the Hubble constant. However, the total density of matter (luminous and dark) is estimated at

$$
\rho_{\mathrm{m}, 0} \approx 2.8 \cdot 10^{-27} \mathrm{~kg} \mathrm{~m}^{-3}
$$

To resolve this discrepancy, the standard cosmological model assumes that the Universe is filled with a mysterious 'dark energy' of constant energy density $E_{\Lambda}$.

Determine the value of $E_{\Lambda}$ and calculate for which redshift in the past the energy density equivalent to matter was equal to the density of dark energy. Neglect the contribution of electromagnetic radiation.\\


\end{document}