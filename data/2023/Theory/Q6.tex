\documentclass[10pt]{article}
\usepackage[utf8]{inputenc}
\usepackage[T1]{fontenc}
\usepackage{amsmath}
\usepackage{amsfonts}
\usepackage{amssymb}
\usepackage[version=4]{mhchem}
\usepackage{stmaryrd}
\usepackage{caption}
\usepackage{multirow}

\DeclareUnicodeCharacter{00D7}{\ifmmode\times\else{$\times$}\fi}

\begin{document}

\section*{Theory 6: 'Bolometer'}
The entrance cavity of a particular bolometer is a cone with an opening angle of $30^{\circ}$, the surface of which has an energy absorption coefficient of $a=0.99$. Assume that there is no scattering of the incident radiation on the walls of the cavity, only multiple specular reflections. The bolometer is connected to a cooler which keeps the bolometer cavity surface at practically 0 K temperature. The instrument is orbiting at 2 au from the Sun and is pointed directly at the centre of the Solar disk.

Calculate the temperature of a black body which would radiate the same amount of energy from a unit surface area as the bolometer opening does.

Note: the opening angle is defined as twice the angle between the axis of the cone and its generatrix.\\

\end{document}