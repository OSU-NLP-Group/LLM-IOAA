\documentclass[10pt]{article}
\usepackage[utf8]{inputenc}
\usepackage[T1]{fontenc}
\usepackage{amsmath}
\usepackage{amsfonts}
\usepackage{amssymb}
\usepackage[version=4]{mhchem}
\usepackage{stmaryrd}
\usepackage{caption}
\usepackage{multirow}

\DeclareUnicodeCharacter{00D7}{\ifmmode\times\else{$\times$}\fi}

\begin{document}

\section*{Theory 4: 'Europa'}
(a) Assuming that the ice covering the ocean on Jupiter's moon Europa is 6 km thick, that the surface temperature on the night side of Europa is 100 K and that the temperature at the ice-water boundary is 273 K , calculate the total power corresponding to the heat emitted from the interior of this moon.\\
(b) On Earth, the geothermal heat flux measured at the surface is $70 \times 10^{-3} \mathrm{Wm}^{-2}$ and originates mainly from radioactive decay. Is the heat emanating from the interior of Europa more likely to come from radioactive decay or tidal forces? (Select the correct answer on the answer sheet and show your working.)\\
(10 points)

Notes: the heat passing through a wall with a surface $S$ and thickness $d$ in time $t$ is described by the formula:

$$
Q=\lambda S \Delta T t / d
$$

where $\lambda$ stands for thermal conductivity and $\Delta T$ for the temperature difference.\\
The thermal conductivity of ice $\lambda=3 \mathrm{Wm}^{-1} \mathrm{~K}^{-1}$. The mass and radius of Europa are $4.8 \times 10^{22} \mathrm{~kg}$ and 1561 km .


\end{document}