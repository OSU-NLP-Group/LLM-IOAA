\documentclass[10pt]{article}
\usepackage[utf8]{inputenc}
\usepackage[T1]{fontenc}
\usepackage{amsmath}
\usepackage{amsfonts}
\usepackage{amssymb}
\usepackage[version=4]{mhchem}
\usepackage{stmaryrd}
\usepackage{caption}
\usepackage{multirow}

\DeclareUnicodeCharacter{00D7}{\ifmmode\times\else{$\times$}\fi}

\begin{document}

\section*{Theory 8: 'Neutrinos'}
In a simplified model of a supernova explosion, the core of a star, composed of pure iron ${ }_{26}^{56} \mathrm{Fe}$ nuclei with a total mass of $1 M_{\odot}$, changes into a neutron star composed of individual electrons, protons and neutrons in numerical proportions of $1: 1: 8$. This process is called 'neutronization' and results in the emission of a large number of neutrinos.

How much larger would the flux of neutrinos observed on Earth from the supernova be than the steady neutrino emission of the Sun, if the supernova exploded in the centre of the Galaxy and the process of neutronization of the core took about 0.01 s ? Give an order-of-magnitude answer.\\


\end{document}