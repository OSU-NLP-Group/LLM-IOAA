\documentclass[10pt]{article}
\usepackage[utf8]{inputenc}
\usepackage[T1]{fontenc}
\usepackage{amsmath}
\usepackage{amsfonts}
\usepackage{amssymb}
\usepackage[version=4]{mhchem}
\usepackage{stmaryrd}
\usepackage{caption}
\usepackage{multirow}

\DeclareUnicodeCharacter{00D7}{\ifmmode\times\else{$\times$}\fi}

\begin{document}

\section*{Theory 3: 'Microlensing'}
A faint subdwarf star ( $I=20.4 \mathrm{mag}$ ) in the Galactic bulge was observed to brighten to $I^{\prime}=$ 15.2 mag as a result of gravitational microlensing, allowing a high-resolution spectrum to be obtained with the UVES spectrograph on the Very Large Telescope (mirror diameter 8.2 m ).

Estimate the diameter of the telescope needed to obtain a spectrum of the same quality with the same instrument and exposure time for this star at its normal apparent brightness.\\


\end{document}