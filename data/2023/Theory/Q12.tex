\documentclass[10pt]{article}
\usepackage[utf8]{inputenc}
\usepackage[T1]{fontenc}
\usepackage{amsmath}
\usepackage{amsfonts}
\usepackage{amssymb}
\usepackage[version=4]{mhchem}
\usepackage{stmaryrd}
\usepackage{caption}
\usepackage{multirow}

\DeclareUnicodeCharacter{00D7}{\ifmmode\times\else{$\times$}\fi}

\begin{document}

\section*{Theory 12: 'DART'}
The Double Asteroid Redirection Test (DART) was a NASA mission to evaluate a method of planetary defense against near-Earth objects. The spacecraft hit Dimorphos, a moon of the asteroid Didymos, to study how the impact affected its orbit.\\
(a) Calculate the expected orbital period change (in minutes), assuming that the collision was head-on, central, and perfectly inelastic.

Assume that before the impact Dimorphos orbited Didymos on a circular orbit with a period of $P=11.92 \mathrm{~h}$. The masses of Dimorphos and Didymos are $m=4.3 \times 10^{9} \mathrm{~kg}$ and $M=5.6 \times 10^{11} \mathrm{~kg}$, respectively. The mass and speed of the DART spacecraft relative to Dimorphos at a moment of impact were $m_{\mathrm{s}}=580 \mathrm{~kg}$ and $v_{\mathrm{s}}=6.1 \mathrm{~km} \mathrm{~s}^{-1}$. Neglect the gravitational influence of other bodies.\\
(20 points)\\
(b) In reality, the orbital period of Dimorphos was observed to be changed by $\Delta P_{0}=-33 \mathrm{~min}$. This is due to the momentum transfer associated with the recoil of the ejected debris: the spacecraft was absorbed by the asteroid, but the impact excavated some material from the asteroid and ejected it into space. Calculate the momentum of the ejected debris and express it as a fraction of the momentum of Dimorphos before the collision. You can assume that the mass of the ejected material is much smaller than the mass of Dimorphos.\\
(15 points)\\
(c) Calculate the velocity change (in $\mathrm{mm} \mathrm{s}^{-1}$ ) of Dimorphos as a result of the impact, taking into account the effect of the ejected debris.\\
(5 points)\\

\end{document}