\documentclass[10pt]{article}
\usepackage[utf8]{inputenc}
\usepackage[T1]{fontenc}
\usepackage{amsmath}
\usepackage{amsfonts}
\usepackage{amssymb}
\usepackage[version=4]{mhchem}
\usepackage{stmaryrd}

\begin{document}

\section*{D1. Photometric comparison of surveys ( $\mathbf{7 5}$ points)}
You are an astronomer working with large photometric surveys, such as the Sloan Digital Sky Survey (SDSS) and the Dark Energy Survey (DES), both of which have your host, Observatório Nacional, as a participant. SDSS used a 2.5 m telescope in Apache Point, USA, during the 2000s, and DES used a 4 m telescope in Cerro Tololo, Chile, from 2013 to 2019. Even though they mostly covered different hemispheres of the sky, they had an equatorial region in common known as Stripe 82 that you can use to compare and calibrate the photometry of different data sets, like SDSS and DES.

The following tables containing object positions and magnitudes from Stripe 82 were downloaded for analysis. However, due to a file system corruption on the computer, the file names were scrambled, and now you cannot tell which table belongs to which survey.

Tables 1 and 2 appear next to each other below, with an identification number for each source, its equatorial coordinates, and its magnitude in the $g$-band ( $m_{g}$ ) with its error (err $m_{g}$ ).\\
(a) (5 points) From these tables, which survey (SDSS or DES) is Table 1 and which is Table 2? Assume that both surveys are equivalent regarding detector response, exposure times, and site characteristics.\\
(b) ( 35 points) Using the data in the table, plot the magnitude ( $m_{g}$ ) on the $x$-axis (linear scale) and the error in magnitude (err $m_{g}$ ) on the $y$-axis (logarithmic scale) using the semi-log paper marked as Graph 1. Estimate the angular coefficient A (slope) and linear coefficient B ( $y$-axis intercept) for each dataset. There is no need to calculate the associated errors.\\
(c) (5 points) The Signal to Noise ratio ( $S / N$ ) is approximately the inverse of the error in the magnitude, $S / N \approx 1 /\left(\right.$ err $\left.m_{g}\right)$. Using the linear fit calculated in the previous part, what is the $S / N$ reached for each survey at a magnitude of $m_{g}=21.5 \mathrm{mag}$ ?\\
(d) ( 15 points) An object in Table 1 that is within 1 arcsecond of an object in Table 2 can be considered to be the same object. By looking at the RA and Dec of the objects in both tables, identify the objects in common and write down a new table with the matching IDs, $I D_{1}$ and $I D_{2}$.\\
(e) (15 points) Using the matched table from part (d), plot the $g$-band magnitude of each survey against the other, Table 1 on $x$-axis, and Table 2 on $y$-axis using the millimetre (linear) paper marked as Graph 2. Draw on error bars for each point in both horizontal and vertical directions, using values double err $m_{g}$ (known as a $2 \sigma$ uncertainty). From your graph, identify the stars that would be suitable for photometric calibration between the two surveys and write down their correspondings IDs from Table 1.

\begin{center}
\begin{tabular}{|l|l|l|l|l|l|l|l|l|l|}
\hline
\multicolumn{5}{|c|}{Table 1} & \multicolumn{5}{|c|}{Table 2} \\
\hline
$I D_{1}$ & RA & Dec & $m_{g}$ & err $m_{g}$ & $I D_{2}$ & RA & Dec & $m_{g}$ & err $m_{g}$ \\
\hline
 & (deg) & (deg) & (mag) & (mag) &  & (deg) & (deg) & (mag) & (mag) \\
\hline
1 & 0.047255 & 0.000406 & 21.7649 & 0.0120 & 1 & 0.006167 & 0.066874 & 21.9020 & 0.0576 \\
\hline
2 & 0.064741 & 0.021568 & 21.1111 & 0.0067 & 2 & 0.018660 & 0.007450 & 21.8039 & 0.0529 \\
\hline
3 & 0.064911 & 0.026395 & 21.3931 & 0.0084 & 3 & 0.047853 & 0.061487 & 21.3007 & 0.0418 \\
\hline
4 & 0.098343 & 0.054871 & 21.3934 & 0.0088 & 4 & 0.050870 & 0.015659 & 21.1678 & 0.0388 \\
\hline
5 & 0.022256 & 0.039129 & 21.9933 & 0.0157 & 5 & 0.051270 & 0.020812 & 21.2524 & 0.0401 \\
\hline
6 & 0.006188 & 0.066928 & 21.5490 & 0.0088 & 6 & 0.057414 & 0.075999 & 21.8884 & 0.0578 \\
\hline
7 & 0.083945 & 0.074259 & 21.9395 & 0.0126 & 7 & 0.064745 & 0.021583 & 21.3634 & 0.0422 \\
\hline
8 & 0.076715 & 0.079496 & 21.4808 & 0.0089 & 8 & 0.064910 & 0.026419 & 21.6428 & 0.0488 \\
\hline
9 & 0.057422 & 0.076006 & 21.8897 & 0.0127 & 9 & 0.071102 & 0.091058 & 21.9259 & 0.0751 \\
\hline
10 & 0.024412 & 0.087688 & 21.8341 & 0.0126 & 10 & 0.074946 & 0.002792 & 21.3258 & 0.0410 \\
\hline
11 & 0.044723 & 0.091782 & 21.8868 & 0.0172 & 11 & 0.076709 & 0.079474 & 21.5303 & 0.0476 \\
\hline
12 & 0.071089 & 0.091053 & 21.4390 & 0.0098 & 12 & 0.092635 & 0.077395 & 21.6995 & 0.0513 \\
\hline
 &  &  &  &  & 13 & 0.098343 & 0.054854 & 21.6542 & 0.0499 \\
\hline
 &  &  &  &  & 14 & 0.099332 & 0.093711 & 21.8802 & 0.0577 \\
\hline
\end{tabular}
\end{center}

\end{document}