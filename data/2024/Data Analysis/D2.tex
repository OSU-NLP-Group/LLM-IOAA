\documentclass[10pt]{article}
\usepackage[utf8]{inputenc}
\usepackage[T1]{fontenc}
\usepackage{amsmath}
\usepackage{amsfonts}
\usepackage{amssymb}
\usepackage[version=4]{mhchem}
\usepackage{stmaryrd}

\begin{document}

\section*{D2. Shapley Hypothesis (75 points)}
Globular clusters are one of the oldest components of galaxies. About a century ago, Harlow Shapley studied the distribution of globular clusters in the Milky Way in order to determine the distance from the Sun to the Galactic Centre, with the hypothesis that globular clusters were symmetrically distributed around the Galactic Centre. The table below shows the positions and distance moduli of a few known globular clusters in the Milky Way. The first three columns in the table show the cluster name, galactic longitude (I), and galactic latitude (b). The fourth column shows the distance modulus (i.e. the difference between the apparent and absolute magnitude), for which the values are extinction-corrected. Based on the data in the table:

\begin{center}
\begin{tabular}{|l|l|l|l|}
\hline
Name & $I$ (degrees) & $b$ (degrees) & Distance modulus (mag) \\
\hline
NGC 6522 & 1.025 & -3.926 & 14.3 \\
\hline
NGC 6401 & 3.450 & 3.980 & 14.4 \\
\hline
NGC 6342 & 4.898 & 9.725 & 14.5 \\
\hline
NGC 6553 & 5.253 & -3.029 & 13.6 \\
\hline
NGC 6440 & 7.729 & 3.801 & 14.6 \\
\hline
Ter 12 & 8.358 & -2.101 & 13.6 \\
\hline
VW-CL160 & 10.151 & 0.302 & 14.2 \\
\hline
2MASS-GC01 & 10.471 & 0.100 & 12.6 \\
\hline
NGC 6517 & 19.225 & 6.762 & 14.8 \\
\hline
NGC 6402 & 21.324 & 14.804 & 14.8 \\
\hline
NGC 6712 & 25.354 & -4.318 & 14.3 \\
\hline
NGC 6426 & 28.087 & 16.234 & 16.6 \\
\hline
NGC 5466 & 42.150 & 73.592 & 16.0 \\
\hline
NGC 7089 & 53.371 & -35.770 & 15.3 \\
\hline
NGC 288 & 151.285 & -89.380 & 14.8 \\
\hline
NGC 2298 & 245.629 & -16.006 & 15.0 \\
\hline
NGC 4590 & 299.626 & 36.051 & 15.1 \\
\hline
NGC 4372 & 300.993 & -9.884 & 13.8 \\
\hline
NGC 362 & 301.533 & -46.247 & 14.7 \\
\hline
BH 140 & 303.171 & -4.307 & 13.4 \\
\hline
NGC 5927 & 326.604 & 4.860 & 14.6 \\
\hline
Patchick 126 & 340.381 & -3.826 & 14.5 \\
\hline
NGC 5897 & 342.946 & 30.294 & 15.5 \\
\hline
NGC 6380 & 350.182 & -3.422 & 14.9 \\
\hline
Djor 1 & 356.675 & -2.484 & 15.0 \\
\hline
\end{tabular}
\end{center}

(a) ( $\mathbf{2 5}$ points) Calculate the distance (in parsecs) of each globular cluster from the Sun as well as their Cartesian coordinates $(x, y, z)$. The $x$-axis points to the Galactic Centre and the $y$-axis points in the direction of galactic rotation. The system is right-handed.\\
(b) ( 15 points) From the given data, estimate the distance from the Sun to the centre of the distribution of globular clusters and its uncertainty.\\
(c) (30 points) To test the validity of Shapley's hypothesis that globular clusters are symmetrically distributed around the Galactic Centre, make histograms with five bins (i.e. sort the data and divide them into five equally-sized intervals) for each of the distributions in the $x, y$, and $z$ directions. Mark the value of the quartiles ( $Q_{1}, Q_{2}, Q_{3}$ ) of the three distributions with solid lines on the histograms.

Hint: The three quartiles divide the sorted sample into four sections, each containing $25 \%$ of the data, with the second and third sections representing the interquartile range.\\
(d) (5 points) Using the quartiles, calculate the symmetry factor value for the three distributions as given by:

$$
\Phi_{x}=\frac{\left|Q_{1, x}+Q_{3, x}-2 Q_{2, x}\right|}{Q_{3, x}-Q_{1, x}}, \Phi_{y}=\frac{\left|Q_{1, y}+Q_{3, y}-2 Q_{2, y}\right|}{Q_{3, y}-Q_{1, y}}, \Phi_{z}=\frac{\left|Q_{1, z}+Q_{3, z}-2 Q_{2, z}\right|}{Q_{3, z}-Q_{1, z}}
$$

Classify the three distributions in the $x, y$, and $z$ directions based on their calculated symmetry factor values, according to the table shown below. Hence, on the answer sheet, write True (T) if the analysed sample follows Shapley's hypothesis or False (F) otherwise.

\begin{center}
\begin{tabular}{|l|l|}
\hline
Symmetry factor value & Symmetry type \\
\hline
$0.0 \leq \Phi \leq 0.1$ & symmetrical \\
\hline
$0.1<\Phi \leq 0.2$ & quasi-symmetrical \\
\hline
$\Phi>0.2$ & asymmetrical \\
\hline
\end{tabular}
\end{center}


\end{document}