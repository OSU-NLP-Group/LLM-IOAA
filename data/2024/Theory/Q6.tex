An astronomer takes pictures, in the V-band, of a faint celestial target, from a place with no light pollution. The selected target is the globular cluster Palomar 4, which has an angular diameter of $\theta = \ang{;;72.0}$ and a uniform surface brightness in the V-band of $m_V = \SI{20.6}{mag \per arcsec\squared}$. The observation equipment consists of one {reflector} telescope, with diameter $D = \SI{305}{\milli\m}$ and F-ratio {$f/5$}, and a prime focus CCD with quantum efficiency $\eta = 80\%$ and {square pixels with} size $\ell = \SI{3.80}{\mu\m}$.

Given data:
\begin{itemize}
    \item V-band central wavelength: $\lambda_V = \SI{550}{\nano\m}$
    \item V-band bandwidth: $\Delta \lambda_V = \SI{88.0}{\nano\m}$
    \item Photons flux for a 0-magnitude object in the V-band: \SI{10000}{ counts \per \nano\m \per \centi \m \squared \per \s)}
\end{itemize}

\begin{parts}

    \part[3] Calculate the plate scale (the angle of sky projected per unit length of the sensor) of the observation equipment in \si{arcmin \per \milli\m}.

    \part[4] Estimate the number of pixels, $n_p$, covered by the cluster image on the CCD.
    
    \part[13] With an exposure time of t = \SI{15}{\s}, the astronomer obtains a signal-to-noise ratio of $S/N = 225$. Compute the brightness of the sky at the observation site, knowing that the CCD has a readout noise {(standard deviation)} of \SI{5}{counts \per pixel} and dark noise of \SI{6}{counts \per pixel \per minute}. Give your answer in \si{mag \per arcsec\squared}. {You may find useful: $\sigma_{RON}^2 = n_p \cdot 1 \cdot RON^2$ and $\sigma_{DN}^2 = n_p \cdot DN \cdot t$.}

\end{parts}