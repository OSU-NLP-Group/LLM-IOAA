A peculiar asteroid of mass, $m$, was spotted at a distance, $d$, from a star with mass, $M$. The magnitude of the asteroid's velocity at the time of the observation was $v = \sqrt{\frac{GM}{d}}$, where $G$ is the universal gravitational constant. The distance $d$ is much larger than the radius of the star.

For both of the following items, express your answers in terms of $M$, $d$, and physical or mathematical constants.

\begin{parts}

    \part[8] If the asteroid is initially moving exactly towards the star, how long will it take for it to collide with the star?

    \part[2] If the asteroid is instead initially moving exactly away from the star, how long will it now take for it to collide with the star?

\end{parts}