The accretion of matter onto compact objects, such as neutron stars and black holes, is one of the most efficient ways to produce radiant energy in astrophysical systems. Consider an element of gas of mass $\Delta m$ in a {stationary and geometrically thin} disc of matter with a maximum radius of $R_{max}$ and minimum {stable orbital} radius of $R_{min}$ (with $R_{min}/R_{max} \ll 1$), in rotation around a compact object of mass $M$ and radius $R$.

\begin{parts}
    \part[6] Assuming that an element of gas in the disc follows an approximately Keplerian circular orbit, find the expression for the total mechanical energy per unit mass $\frac{\Delta E}{\Delta m}$ released by this gas from the moment it starts orbiting at a radius $R_{max}$ until the moment it reaches an orbit of radius $r \ll R_{max}$. This process occurs very slowly, transforming kinetic energy into internal energy of the gas disc through viscous dissipation.\\ \textbf{Note:} Ignore the gravitational interaction between particles within the accretion disc and give your final answer in terms of $G$, $M$ and $r$.

    \part[5] Considering that the disc {receives mass} at an average rate of $\dot{M}$, and assuming that all the {mechanical energy lost} is ultimately converted into radiation, {find an expression for the total luminosity of the disc}.


    \part[8] {Consider now the ring composed of all mass elements from radius between $r + \Delta r$ to $r$. In this scenario, find an expression of the luminosity generated by the disc over its small length $\Delta r$  at this radius, that is, find the expression for $\frac{\Delta E}{\Delta t \Delta r}$.}


    \part[10] Assuming that the gravitational energy released in this ring is locally emitted by the surface of the ring in the form of black-body radiation, find an expression for the surface temperature $T$ of the ring.

    \part[3] Consider that the central object is a stellar black hole with a mass of $3M_{\odot}$ and a rate of accretion of  ${\dot M} = 10^{-9}\;M_{\odot}/\mathrm{year}$. Consider also that $R_{min} = 3R_{sch}$, where $R_{sch}$ is the Schwarzschild radius of the black hole. Determine the luminosity of the disc and the peak wavelength of emission of its innermost part. Ignore gravitational redshift effects and assume that the emission from the innermost part of the ring dominates the total emission.

    \part[3] Now, considering another accretion system with $\dot{M} = 1\;M_{\odot}/\mathrm{year}$ and a peak emission wavelength of $\lambda = \SI{6e-8}{\m}$, estimate the mass of this black hole.
    
\end{parts}