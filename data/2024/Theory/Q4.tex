The structure of a white dwarf is sustained against gravitational collapse by the pressure of degenerate electrons, a phenomenon explained by quantum physics and related to the Pauli Exclusion Principle for electrons. The equation of state of a gas made of non-relativistic degenerate electrons is the following:
\[P = \left(\dfrac{3}{8\pi}\right)^{2/3}\dfrac{h^{2}}{5m_{e}}n_{e}^{5/3}\text{,}\]
where $n_e$ is the number of electrons per unit volume, which can be expressed in terms of the mass density $\rho$ using the dimensionless factor $\mu_e$, the number of nucleons (protons and neutrons) per unit electron. Also consider that the central pressure can be described by this equation of state.

In the condition of hydrostatic equilibrium, the pressure and gravitational forces balance each other at any distance $r$ from the centre of the star. This condition can be expressed by:
\[\dfrac{{\rm d} P}{{\rm d} r} = -\dfrac{GM(r)\rho(r)}{r^{2}}\text{,}\]
where $M(r)$ is the mass contained in the sphere of radius $r$, and $\rho(r)$ is the mass density of the star at a radius $r$.

Assume that $m_p = m_n$, the density of a white dwarf is roughly uniform, and the following approximation is valid at the surface of the star:

\[\left.\dfrac{{\rm d} P}{{\rm d} r} \right |_{r = R} \approx -\dfrac{P_{c}}{R}\text{,}\]

where $P_{c}$ is the pressure at the center of the star, and $R$ the star radius.

\begin{parts}

    \part[6] 
    The relationship between the mass $M$ and the radius $R$ of a white dwarf can be written in the form:
    
    $$R = a \cdot M^{b}$$
    
    Find the exponent $b$ and determine the coefficient $a$ in terms of physical constants and $\mu_e$. 

    \part[4] Using the relationship found in the previous part, estimate the radius of a white dwarf made of fully ionised carbon ($^{12}_6 \mathrm{C}$) with a mass of $M = 1.0 \, M_{\odot}$.
    
\end{parts}